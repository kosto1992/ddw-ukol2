\documentclass[11pt]{beamer}
\usetheme{Warsaw}
\usepackage[utf8]{inputenc}
\usepackage[slovak]{babel}
\usepackage[T1]{fontenc}

\author{Lukáš Koštenský}
\title{Získanie pravidiel pre odporúčanie produktov}
%\setbeamercovered{transparent} 
%\setbeamertemplate{navigation symbols}{} 
%\logo{} 
\institute{MI-DDW} 
%\date{} 
\begin{document}

\begin{frame}
\titlepage
\end{frame}

%\begin{frame}
%\tableofcontents
%\end{frame}

\begin{frame}{Popis aplikácie}

\section{Dataset}
Použil som vzorový dataset obsahujúci informácie o obsahu nákupneho košíku nakupujúceho. Tento dataset je dostupný na internetovej stránke http://www.sevana.fi/sample1.xls

\section{Predspracovanie}

RapidMiner po načítaní Excel súboru použije:

„Nominal to Binominal“, ktorý prevedie atribúty s viac hodnotami ako dvoma do binárnych hodnôt. To je potrebné spraviť, pretože asociačné pravidlá pracujú len s binárnymi atribútmi.

„FP-Growth“ - predzpracuje dáta, musel som znížiť min-support, resp. vynútiť minimálny počet výsledkov.

„Create Association Rules“ generuje asociačné pravidla.
\end{frame}
\begin{frame}{Výsledky a použité nástroje}

\section{Výsledky}
Výsledkom spracovania je súbor pravidiel. Podľa toho, čo už zákazník nakúpil, vieme určiť čo bude chcieť k týmto veciam dokúpiť.

Pr.
Association Rules
[mouse = USB cable] --> [DVD ROM = keyboard] (confidence: 1.000)
[keyboard = MP3 player] --> [DVD ROM = keyboard] (confidence: 1.000)
...

Použité nástroje:
\begin{enumerate}
\item RapidMiner 
\end{enumerate}
\end{frame}

\end{document}